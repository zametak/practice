%------------------Settings-------------------------

\documentclass[12pt]{article}
\usepackage[utf8]{inputenc}
\usepackage[russian]{babel}
\usepackage{amsmath,amssymb}
\usepackage{graphics}
\usepackage{pbox}
\usepackage[x11names]{xcolor}
\definecolor{brightmaroon}{rgb}{0.76, 0.13, 0.28}
\definecolor{royalazure}{rgb}{0.0, 0.22, 0.66}
\usepackage[colorlinks=true,linkcolor=royalazure]{hyperref}
\usepackage{tikz, tkz-fct, pgfplots}
\usetikzlibrary{arrows}
\usepackage{geometry}
\geometry{
	a4paper,
	total={170mm,257mm},
	left=20mm,
	top=20mm
} 
\usepackage[labelsep=period]{caption}

% ----------------- Commands ----------------- 

\newcommand{\eps}{\varepsilon}
\newcommand\tline[2]{$\underset{\text{#1}}{\text{\underline{\hspace{#2}}}}$}

% ----------------- Set graphics path ----------------- 
\graphicspath{{img/}}
\begin{document}
\pagestyle{empty}

% ----------------------Title----------------------------------
\centerline{\large Министерство науки и высшего образования}	
\centerline{\large Федеральное государственное бюджетное образовательное}
\centerline{\large учреждение высшего образования}
\centerline{\large ``Московский государственный технический университет}
\centerline{\large имени Н.Э. Баумана}
\centerline{\large (национальный исследовательский университет)''}
\centerline{\large (МГТУ им. Н.Э. Баумана)}
\hrule
\vspace{0.5cm}
\begin{figure}[h]
\center
\includegraphics[height=0.35\linewidth]{bmstu-logo-small.png}
\end{figure}
\begin{center}
	\large	
	\begin{tabular}{c}
		Факультет ``Фундаментальные науки'' \\
		Кафедра ``Высшая математика''		
	\end{tabular}
\end{center}
\vspace{0.5cm}
\begin{center}
	\LARGE \bf	
	\begin{tabular}{c}
		\textsc{Отчёт} \\
		по учебной практике \\
		за 1 семестр 2020---2021 гг.
	\end{tabular}
\end{center}
\vspace{0.5cm}
\begin{center}
	\large
	\begin{tabular}{p{5.3cm}ll}
		\pbox{5.45cm}{
			Руководитель практики,\\
			ст. преп. кафедры ФН1} 	& \tline{\it(подпись)}{5cm} & Кравченко О.В. \\[0.5cm]
		студент группы ФН1--11 		& \tline{\it(подпись)}{5cm} & Аверина А.А.
	\end{tabular}
\end{center}
\vfill
\begin{center}
	\large	
	\begin{tabular}{c}
		Москва, \\
		2020 г.
	\end{tabular}
\end{center}
 \newpage
 \newpage	
\tableofcontents

%------------------Table of contents----------------------
\newpage
\section{Цели и задачи практики}	
\subsection{Цели}
--- развитие компетенций, способствующих успешному освоению материала бакалавриата и необходимых в будущей профессиональной деятельности.

\subsection{Задачи}
\begin{enumerate}
\item Знакомство с программными средствами, необходимыми в будущей профессиональной деятельности.
\item Развитие умения поиска необходимой информации в специальной литературе и других источниках.
\item Развитие навыков составления отчётов и презентации результатов.
\end{enumerate}

\subsection{Индивидуальное задание}	
\begin{enumerate}
\item Изучить способы отображения математической информации в системе вёртски \LaTeX.
\item Изучить возможности  системы контроля версий \textsf{Git}.
\item Научиться верстать математические тексты, содержащие формулы и графики в системе \LaTeX.
Для этого, выполнить установку свободно распространяемого дистрибутива \textsf{TeXLive} и оболочки \textsf{TeXStudio}.
\item Оформить в системе \LaTeX типовые расчёты по курсе математического анализа согласно своему варианту.
\item Создать аккаунт на онлайн ресурсе \textsf{GitHub} и загрузить исходные \textsf{tex}--файлы 
и результат компиляции в формате \textsf{pdf}.
\end{enumerate} 

%---------------------------------------------------------------
\newpage
\section{Отчёт}
Актуальность темы продиктована необходимостью владеть системой вёрстки \LaTeX и средой вёрстки \textsf{TeXStudio} для
отображения текста, формул и графиков. Полученные в ходе практики навыки могут быть применены при написании
курсовых проектов и дипломной работы, а также в дальнейшей профессиональной деятельности.

Ситема вёрстки \LaTeX содержит большое количество инструментов (пакетов), упрощающих отображение информации в различных 
сферах инженерной и научной деятельности. 
%-----------------------------------------------------------------
\newpage
\section{Индивидуальное задание}
%\subsection{Элементарные функции и их графики.}
%\input{src/part1.tex}

%==============================================================================
\subsection{Пределы и непрерывность.}

%---------------------------- Problem 1----------------------------------
\subsubsection*{\center Задача № 1.}
{\bf Условие.~}
Дана последовательность $a_{n}=\frac{7n+4}{2n+1}$ и число $c=\frac{7}{2}$. Доказать, что $\lim\limits_{x\rightarrow\infty} a_{n}=c $, а именно, для каждого $\varepsilon>0$ найти наименьшее натуральное число  $N{=}N(\varepsilon)$ такое, что $|a_{n}-c|<\varepsilon$ для всех $n>N(\varepsilon)$. Заполнить таблицу: 
\begin{center}
\begin{tabular}{ | p{25pt} | c | c | c | c |}
\hline
$\varepsilon$& $0{,}1$ & $0{,}01$ & $0{,}001$ \\ \hline
$N(\varepsilon)$ &   &   &\\
\hline
\end{tabular}
\end{center}
\medskip
%=====================================================================
{\bf Решение.~}
Рассмотрим неравенство $a_{n}-c<\varepsilon$, $\forall\varepsilon>0$, учитывая выражение для $a_{n}$ и $c$ из условия варианта, получим 
$$\left|\frac{7n+4}{2n+1}-\frac{7}{2}\right|<\varepsilon$$
Неравенство запишем в виде двойного неравенства и приведём выражение под знаком модуля к общему знаменателю, получим
$${-}\varepsilon <\frac{1}{4n+2}<\varepsilon$$
Заметим, что левое неравенство выполнено для любого номера $n\in \mathbb{N}$ поэтому, будем рассматривать правое неравенство
$$\frac{1}{4n+2}<\varepsilon$$
Выполнив цепочку преобразований, перепишем неравенство относительно $n$, и, учитывая, что $n\in \mathbb{N}$, получим 
$$\frac{1}{2(2n+1)}<\varepsilon,$$
$$2n+1>\frac{1}{2\varepsilon},$$
$$n>\frac{1}{4\varepsilon}-\frac{1}{2},$$
$$n>\frac{1-2\varepsilon}{4\varepsilon},$$
$$N(\varepsilon)=\biggl[\frac{1-2\varepsilon}{4\varepsilon}\biggr],$$
где $[\;]$ -- целая часть от числа. Заполним таблицу:
\begin{center}
\begin{tabular}{ | p{25pt} | c | c | c | c |}
\hline
$\varepsilon$& $0{,}1$ & $0{,}01$ & $0{,}001$ \\ \hline
$N(\varepsilon)$ & 2  & 24 & 249\\
\hline
\end{tabular}
\end{center}
{\bf Проверка:~}
$$|a_{3}-c|=\frac{1}{15}<0{,}1,$$
$$|a_{25}-c|=\frac{1}{102}<0{,}01,$$
$$|a_{250}-c|=\frac{1}{1002}<0{,}001.$$
\newpage
% ---------------------------- Problem 2----------------------------------
\subsubsection*{\center Задача № 2.}
{\bf Условие.~}
Вычислить пределы функций
$$
\begin{array}{cc}
\text{\bf(а):} &  \lim\limits_{x\rightarrow-1}\dfrac{(x^3-2x-1)^2}{x^4+2x+1} , \\[10pt]
\text{\bf(б):} & \lim\limits_{x\rightarrow+\infty} \dfrac{\sqrt[3]{x^2\sqrt{x^2+1}}+3x}{x} ,\\[10pt]
\text{\bf(в):} & \lim\limits_{x\rightarrow0} \dfrac{\sqrt{x+1}\sqrt{1+2x}-1}{x},\\[10pt]
\text{\bf(г):} & \lim\limits_{x\rightarrow+0} \biggl( 2-e^{arcsin^2\sqrt{x}}\biggl)^{3/x}, \\[10pt]
\text{\bf(д):} & \lim\limits_{x\rightarrow0} \left(\dfrac{tg4x}{x}\right)^{2+x} , \\[10pt]
\text{\bf(е):}  & \lim\limits_{x\rightarrow-\pi} \dfrac{sin4x}{x^2+\pi x} . \\
\end{array}
$$
\\
{\bf Решение.~}\\
\\
\text{\bf(а):}
$$
\begin{array}{l}
\lim\limits_{x\rightarrow-1} \dfrac{(x^3-2x-1)^2}{x^4+2x+1} =  \lim\limits_{x\rightarrow-1}  \dfrac{(x+1)^2(x^2-x-1)^2}{(x+1)(x^3-x^2+x+1)} = \lim\limits_{x\rightarrow-1}  \dfrac{(x+1)(x^2-x-1)^2}{x^3-x^2+x+1}=0
\end{array}
$$
\\
\text{\bf(б):}
$$
\begin{array}{l}
\lim\limits_{x\rightarrow+\infty} \dfrac{\sqrt[3]{x^2\sqrt{x^2+1}}+3x}{x}=\lim\limits_{x\rightarrow+\infty} \dfrac{\sqrt[3]{x^2\sqrt{x^2+1}}}{x}=\lim\limits_{x\rightarrow+\infty} \sqrt[3]{\frac{x^2}{x^3}\sqrt{x^2+1}}=\lim\limits_{x\rightarrow+\infty} \sqrt[6]{{\frac{x^2+1}{x^2}}}=\\
= \lim\limits_{x\rightarrow+\infty} \sqrt[6]{1+\frac{1}{x^2}}= \lim\limits_{x\rightarrow+\infty} \sqrt[6]{\frac{1}{x^2}}=0
\end{array}
$$
\text{\bf(в):}
 $$
 \begin{array}{l} 
 \lim\limits_{x\rightarrow0} \dfrac{\sqrt{x+1}\sqrt{1+2x}-1}{x} = \left[\dfrac{0}{0} \right]= \lim\limits_{x\rightarrow0} \dfrac{(x+1)(1+2x)-1}{x(\sqrt{x+1}\sqrt{1+2x}+1)} = \lim\limits_{x\rightarrow0} \dfrac{2x+3}{\sqrt{x+1}\sqrt{1+2x}+1}\\ \medskip{}{} =\dfrac{3}{2}
 \end{array}
 $$
\text{\bf(г):}
$$
\begin{array}{l}
\lim\limits_{x\rightarrow+0} \biggl( 2-e^{arcsin^2\sqrt{x}}\biggr)^{3/x}=\lim\limits_{x\rightarrow+0} e ^ \biggl( ln\biggl( 2-e^{arcsin^2\sqrt{x}} \biggr) \biggr)^{3/x}} =e^{\lim\limits_{x\rightarrow+0}\frac{3}{x}ln(2-arcsin^2\sqrt{x})}= \\
=e^{\lim\limits_{x\rightarrow+0}-\frac{3}{x}({arcsin^2\sqrt{x}})}=e^{-\lim\limits_{x\rightarrow+0}\frac{3x}{x}}=e^{-3}
\end{array}
$$
\\
\text{\bf(д):}
$$
\begin{array}{l}
 \lim\limits_{x\rightarrow0} \left(\dfrac{tg4x}{x}\right)^{2+x}=\left| tg4x \sim 4x, x \to 0 \right|=\lim\limits_{x\rightarrow0} \left(\dfrac{4x}{x}\right)^{2+x}=\lim\limits_{x\rightarrow0}4^{2+x}=16
\end{array}
$$
\text{\bf(е):}
$$
\begin{array}{l}
\lim\limits_{x\rightarrow-\pi} \dfrac{sin4x}{x^2+\pi x}=\left| t=x+\pi, t \to 0\right|=\lim\limits_{t\rightarrow0} \dfrac{sin4(t-\pi)}{(t-\pi)^2+\pi (t-\pi)}=\left| sin4(t-\pi)\sim 4(t-\pi) , t \to 0\right| = \\
= \lim\limits_{t\rightarrow0} \dfrac {4} {(t-\pi)+\pi}=\lim\limits_{t\rightarrow0} \dfrac {4} {t}=\infty
\end{array}
$$
% ---------------------------- Problem 3----------------------------------
\subsubsection*{\center Задача № 3.}
{\bf Условие.~}\\
\text{\bf(а):} Показать, что данные функции
$f(x)$ и $g(x)$ являются бесконечно малыми или бесконечно большими
при указанном стремлении аргумента. \\
\text{\bf(б):} Для каждой функции $f(x)$ и $g(x)$ записать главную часть
(эквивалентную ей функцию)  вида $C(x-x_0)^{\alpha}$ при $x\rightarrow x_0$ или $Cx^{\alpha}$
при $x\rightarrow\infty$, указать их порядки малости (роста). \\
\text{\bf(в):} Сравнить функции $f(x)$ и $g(x)$ при указанном стремлении.
\begin{center}
	\begin{tabular}{|c|c|c|}
		\hline
		№ варианта & функции $f(x)$ и $g(x)$ & стремление \\[6pt]
		%\hline
		2 & $f(x) = \frac{\sqrt{x}}{\sqrt[3]{x+1}-\sqrt[3]{x}},~g(x)=x^2ln\frac{3x^2+2x+1}{3x^2}$ & $x\rightarrow+\infty$ \\
		\hline
		\end{tabular}
		\bigskip
		\\
		{\bf Решение.~}\\
		\end{center}
		\medskip
		\text{\bf(а):}~Покажем, что $f(x)$ и $g(x)$ бесконечно большие функции,
$$
 \begin{array}{l} 
\lim\limits_{x\rightarrow+ \infty} f(x)=\lim\limits_{x\rightarrow +\infty} \dfrac{\sqrt{x}}{\sqrt[3]{x+1}-\sqrt[3]{x}}=\lim\limits_{x\rightarrow+\infty} \dfrac{\sqrt{x}}{\sqrt[3]{x}\biggl(\sqrt[3]{1+\frac{1}{x}}-1\biggr)}=\lim\limits_{x\rightarrow +\infty}x^{\frac{1}{6}}\biggl({\biggl(1+\frac{1}{x}}\biggr)^{\frac{2}{3}}+\sqrt[3]{1+\frac{1}{x}}+1  \biggr)=\\ =3\lim\limits_{x\rightarrow+\infty}x^{\frac{1}{6}}=\infty , \\
 
\lim\limits_{x\rightarrow+ \infty} g(x)= \lim\limits_{x\rightarrow \infty}  x^2ln\frac{3x^2+2x+1}{3x^2}=\lim\limits_{x\rightarrow +\infty}x^2ln\biggl(1+\dfrac{2}{3x}+\dfrac{1}{3x^2}\biggr)=\lim\limits_{x\rightarrow +\infty}x^2\biggl(\dfrac{2}{3x}+\dfrac{1}{3x^2}\biggr)=\\
=\lim\limits_{x\rightarrow +\infty}x^2 \biggl(\dfrac{2}{3x}+\dfrac{1}{3x^2}\biggr)=\lim\limits_{x\rightarrow +\infty}\biggl( \dfrac{2}{3x}+\dfrac{1}{3}\biggr)=\lim\limits_{x\rightarrow +\infty} \dfrac{2x}{3}=\infty.
\end{array}
$$
\text{\bf(б):}~Так как $f(x)$ и $g(x)$ бесконечно большие функции, то эквивалентными им будут функции вида 
$Cx^{\alpha}$ при $x\rightarrow+\infty$. Найдём эквивалентную для $f(x)$ из условия
$$
\lim\limits_{x\rightarrow+\infty}\dfrac{f(x)}{x^{\alpha}} = C,
$$
где $C$ --- некоторая константа. Рассмотрим предел
$$
\lim\limits_{x\rightarrow +\infty} \dfrac{\sqrt{x}}{x^{\alpha}\biggl(\sqrt[3]{x+1}-\sqrt[3]{x}\biggr)}=3\lim\limits_{x\rightarrow+\infty}\dfrac{x^{\frac{1}{6}}}{x^{\alpha}}
$$
При $\alpha=\dfrac{1}{6}$ последний предел равен $3$, отсюда $C=3$ и 
$$
f(x)\sim 3x^{\frac{1}{6}}~\text{при}~x\rightarrow+\infty.
$$
Аналогично, рассмотрим предел
$$
\lim\limits_{x\rightarrow+\infty}\dfrac{g(x)}{x^{\alpha}} =\lim\limits_{x\rightarrow+\infty}\dfrac{x^2ln\frac{3x^2+2x+1}{3x^2}}{x^{\alpha}}=\lim\limits_{x\rightarrow +\infty} \dfrac{2x}{3x^{\alpha}}
$$
При $\alpha=1$ последний предел равен $\frac{2}{3}$, отсюда $C=\frac{2}{3}$ и
$$
g(x)\sim\frac{2}{3} x~\text{при}~x\rightarrow+\infty.
$$
\newpage
\text{\bf(в):}~Для сравнения функций $f(x)$ и $g(x)$ рассмотрим предел их отношения при указанном стремлении
$$
\lim\limits_{x\rightarrow\infty}\dfrac{f(x)}{g(x)}.
$$
Применим эквивалентности, определенные в пункте (б), получим
$$
\lim\limits_{x\rightarrow+\infty}\dfrac{f(x)}{g(x)} = \lim\limits_{x\rightarrow+\infty}\dfrac{3x^{\frac{1}{6}}}{\frac{2x}{3}}=\lim\limits_{x\rightarrow+\infty} \dfrac{9}{2x^{\frac{5}{6}}}=0
$$
Отсюда, $g(x)$ есть бесконечно большая функция более высокого порядка роста, чем $f(x)$.
%=================================================================================================================================
%\subsection{Приложения дифференциального исчисления.}
%\input{src/part3.tex}

\newpage
\addcontentsline{toc}{section}{Список литературы}
\begin{thebibliography}{99}
\bibitem{book01} Львовский С.М. Набор и вёрстка в системе \LaTeX, 2003 c.
\bibitem{book02} Е.М. Балдин Компьютерная типография \LaTeX.

\end{thebibliography}
\end{document}
